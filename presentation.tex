\usepackage{fontspec}
\usepackage{polyglossia}
\usepackage{tikz}

\setmainlanguage{english}
\usetheme{cambridge}

\graphicspath{{imgs/}{svgs/}}

% TikZ libraries and styles
\usetikzlibrary{positioning,shapes.geometric,graphs,graphdrawing,quotes,fit}
\usegdlibrary{layered,trees}
\colorlet{GenericItem}{yellow!20!white}
\colorlet{GenericHighlight}{red!20!white}
\tikzset{Term/.style={draw,fill=yellow!20!white,font=\tiny\bfseries}}
\tikzset{Relation/.style={draw=blue!40!white,thick,arrows=<-}}
\tikzset{RelationLabel/.style={shape=circle,fill=white,draw=black,thin,font=\tiny\bfseries,inner sep=1pt}}
\tikzset{File/.style={draw,fill=yellow!20!white,font=\tiny\ttfamily}}
\tikzset{Entity/.style={draw,fill=yellow!20!white,font=\scriptsize}}
\tikzset{Process/.style={draw,shape=ellipse,fill=blue!20!white,font=\scriptsize}}

\title{(Re-)bridging the anatomy ontologies with SSSOM}

\author{Damien~Goutte-Gattat\inst{1}}
\institute{\inst{1}FlyBase group, Department of Physiology, Development and Neuroscience}
\date{International Conference on Biomedical Ontologies 2024, July 2024}

\hypersetup{pdfpagemode=UseNone,
  pdftitle={},
  pdfauthor={Damien Goutte-Gattat}}
\AtBeginSection{
  \begin{frame}{Outline}{}
    \tableofcontents[currentsection]
  \end{frame}
}

\begin{document}

\begin{frame}
  \titlepage
\end{frame}

\section{Introduction}

\begin{frame}
  \frametitle{An approach to a multi-species anatomy ontology}

  \begin{columns}
    \begin{column}{.5\textwidth}
      \begin{tikzpicture}[nodes={shape=circle,draw,fill=GenericItem}]
        \node (core) {Uberon};
        \node (fbbt) at (150:2cm) {FBbt};
        \node (zfa)  at (90:2cm)  {ZFA};
        \node (xao)  at (30:2cm)  {XAO};
        \node (ma)   at (210:2cm) {MA};
        \node (wbbt) at (270:2cm) {WBbt};
        \node (tx1) at  (310:2cm) {...};
        \node (tx2) at  (330:2cm) {...};
        \node (tx3) at  (350:2cm) {...};
        \draw (core) -- (fbbt) (core) -- (zfa)  (core) -- (xao)
              (core) -- (ma)   (core) -- (wbbt) (core) -- (tx1)
              (core) -- (tx2)  (core) -- (tx3);
      \end{tikzpicture}
    \end{column}
    \begin{column}{.45\textwidth}
      Strategy:
      \begin{itemize}
        \item Provide a taxon-neutral core framework
        \item Integrate existing taxon-specific ontologies
      \end{itemize}
    \end{column}
  \end{columns}

  \note{Anatomy ontologies are used in many aspects of biocuration, for
  example to capture the organ of origin of a biological sample, the tissue
  where a gene is expressed, or the anatomical location where a disease
  manifests itself.

  Many species-specific curation projects have, therefore, their own ontology
  to describe the anatomy of their favourite organism. For example, at FlyBase
  we have our own \emph{Drosophila} Anatomy Ontology (FBbt), ZFIN has its
  Zebrafish Anatomy Ontology (ZFA), XenBase has its Xenopus Anatomy Ontology
  (XAO), and so on.

  But to enable \emph{cross-species} curation and analysis of data (especially
  high-throughput data, such as scRNAseq), we need ideally a single,
  multi-species ontology that covers all our model organisms. For the past 12
  years or so, this has been provided by Uberon, the Uber Anatomy Ontology.

  Rather than duplicating the work already done by pre-existing,
  species-specific ontologies, Uberon's strategy is to provide a taxon-neutral
  core framework, describing common anatomical structures in broadly generic
  terms, and to plug the species-specific ontologies into that core.}

\end{frame}

\begin{frame}
  \frametitle{Example: integration with FBbt}

  \only<1>{
    \begin{columns}[t]
      \begin{column}{.45\textwidth}
        Uberon hierarchy
        \begin{tikzpicture}
          \graph[layered layout,orient=down,nodes={Term},edge quotes mid,edges={Relation,nodes={RelationLabel}}] {
            anatomical entity <-
            anatomical structure <-
            circulatory organ <- {
              primary circulatory organ <- { heart, dorsal vessel heart },
              accessory circulatory organ <- lymph heart
            };
          };
        \end{tikzpicture}
      \end{column}
      \begin{column}{.45\textwidth}
        FBbt hierarchy
        \begin{tikzpicture}
          \graph[layered layout,orient=down,nodes={Term},edge quotes mid,edges={Relation,nodes={RelationLabel}}] {
            anatomical entity <-
            anatomical structure <-
            heart <-["P"] {
              aortic funnel, heart chamber, dorsal diaphragm
            };
          };
        \end{tikzpicture}
      \end{column}
    \end{columns}
  }
  \only<2>{
    \begin{columns}
      \begin{column}{.3\textwidth}
        Integrated hierarchy
      \end{column}
      \begin{column}{.6\textwidth}
        \begin{tikzpicture}
          \graph[layered layout,orient=down,nodes={Term},edge quotes mid,edges={Relation,nodes={RelationLabel}}] {
            anatomical entity <-
            anatomical structure <-
            circulatory organ <- {
              primary circulatory organ <- {
                heart,
                dorsal vessel heart <- "heart (drosophila)" <-["P"] {
                  aortic funnel, heart chamber, dorsal diaphragm
                }
              },
              accessory circulatory organ <- lymph heart
            };
          };
        \end{tikzpicture}
      \end{column}
    \end{columns}
  }
  
  \note<1>{To illustrate this strategy, here is (on the left) a simplified
  excerpt of Uberon's hierarchy around the concept of ``heart'', which in
  Uberon parlance is called, more generically, the ``primary circulatory
  organ'' (to make room for the \emph{accessory} circulatory organs that can
  be found in some species).

  Below ``primary circulatory organ'', Uberon defines both ``heart'', which is
  intended to refer to the \emph{vertebrate} heart, and ``dorsal vessel
  heart'', which refers to the heart of arthropods such as fruit flies. Uberon
  itself does not describe the arthropod heart in any more details -- there
  are no terms under ``dorsal vessel heart''. That term is in fact merely
  intended as a ``branching point'', where more specific terms from
  arthropod-specific ontologies can be rattached.

  In the \emph{Drosophila} Anatomy Ontology, we have a single term ``heart''
  to refer to the dorsal vessel heart (we don't need to call it ``dorsal
  vessel heart'' since it is the only heart that exists in flies), and below
  that term, we have a handful of terms that describe more precisely the
  different parts that make up the fly heart.}

  \note<2>{When FBbt is integrated into Uberon, its ``heart'' term is
  automatically treated as a subclass of the more generic ``dorsal vessel
  heart'', thereby creating a unified hierarchy where the
  \emph{Drosophila}-specific terms are appropriately rattached to Uberon's
  taxon-neutral core.

  Using this strategy, Uberon is able to accurately and precisely describe the
  anatomy of the most used model organisms, by benefitting from all the
  existing species-specific ontologies without having to duplicate their
  efforts.

  This talk is about how this integration process works.}

\end{frame}

\begin{frame}
  \frametitle{Integration requires cross-ontology mappings}

  \begin{block}{}
    Integrating the taxon-specific ontologies with the Uberon core framework
    requires to know where taxon-specific terms should be ``plugged'' in the
    core framework.

    E.g., we need to know that FBbt's \emph{heart} must be attached to Uberon's
    \emph{dorsal vessel heart}.
  \end{block}

  \note{The key component of the integration process is the knowledge of which
  species-specific terms, from each of the species-specific ontologies such as
  FBbt, should be ``attached'' to which taxon-neutral term in Uberon's core
  framework. For example, knowing that FBbt's ``heart'' is the fly-specific
  term for what Uberon calls the ``dorsal vessel heart''.

  This is the role of cross-ontology \emph{mappings} -- basically, a mapping
  is a statement that one term from an ontology A is ``mapped'', or
  ``related'', to another term in an ontology B.}

\end{frame}

\section{Current system}

\begin{frame}
  \frametitle{Maintaining mappings as cross-references}

  \begin{block}{}\ttfamily
[Term]

id: UBERON:0015230

name: dorsal vessel heart

def: "The caudal, pulsatile region of the dorsal vessel of the arthropod circulatory system."

is\_a: UBERON:0007100 ! primary circulatory organ

xref: FBbt:00003154
  \end{block}

  Limitations:
  \begin{itemize}
    \item Overloading of the \emph{oboInOwl:hasDbXref} annotation
    \item No metadata
    \item Mappings are embedded within the ontology and must be extracted
  \end{itemize}

  \note{In the current system, mappings are represented and maintained as
  \emph{cross-references}, that is, as a \emph{oboInOwl:hasDbXref} annotation
  that is carried by the taxon-neutral term in Uberon and that points to the
  taxon-specific term in another ontology -- as shown here, where Uberon's
  ``dorsal vessel heart'' term carries a ``xref'' to FBbt:00003154, which is
  FBbt's term for the fly heart that we have seen before.

  That's a pretty simple way of representing mappings, but it has several
  drawbacks.

  First, the \emph{oboInOwl:hasDbXref} annotation has been used for a lot of
  things since its introduction, and no longer has a clear and unambiguous
  meaning -- what it means depends on where it is used.

  Second, all this representation allows us to say is that there is a mapping,
  but we can't know anything else about said mapping. Why does that mapping
  exist? Who asserted it? When? On what basis? This is simply not recorded.

  Third, the mappings are not available separately from the entire ontology.
  This makes reusing those mappings more difficult than it should be
  (difficult to Find, Access and Reuse).}

\end{frame}

\begin{frame}
  \frametitle{From cross-references to bridge files}

  \begin{columns}
    \begin{column}{.3\textwidth}
      \begin{tikzpicture}
        \node [File] (uberon) at (0,0)     {uberon.obo};
        \node [File] (script) at (0,-1)    {make-bridge-ontology-from-xrefs.pl};
        \node [File] (bridge) at (0,-2)    {uberon-bridge-to-fbbt.obo};
        \node [File]          at (0, -2.5) {uberon-bridge-to-wbbt.obo};
        \node [File]          at (0, -3)   {uberon-bridge-to-ma.obo};
        \node [File]          at (0, -3.5) {uberon-bridge-to-....obo}; 
        \draw[->] (uberon) -- (script) (script) -- (bridge);
      \end{tikzpicture}
    \end{column}
    \begin{column}{.68\textwidth}
      \begin{block}{uberon-bridge-to-fbbt.obo}\ttfamily
[Term]

id: FBbt:00003154

intersection\_of: UBERON:0015230

intersection\_of: part\_of NCBITaxon:7227
      \end{block}

      Can be used to integrate FBbt into Uberon
    \end{column}
  \end{columns}

  \medskip
  Limitations:
  \begin{itemize}
    \item Highly Uberon-specific code -- cannot be reused outside of Uberon
    \item Dependent on the OBO format
  \end{itemize}

  \note{Whenever we make a new Uberon release, the mappings (stored as
  cross-references within Uberon) are extracted by a specialised Perl script
  to produce a set of \emph{bridge files} (one for each species-specific
  ontology). The bridge files are a OBO materialisation of the mappings; they
  contain minimalist frames that serve to rattach a species-specific term to a
  core Uberon term, as in this example, which states that FBbt:00003154 (the
  fly heart) is equivalent to a Uberon's ``dorsal vessel heart'' that is part
  of the \emph{Drosophila melanogaster} taxon.

  The main problem with this derivation is that the aforementioned script is
  highly specific to Uberon. Most of its behaviours are hard-coded, and the
  script is virtually impossible to re-use outside of Uberon. This means,
  among other things, that third-party projects that might need those bridges,
  for one reason or another, are entirely dependent on Uberon to produce them.
  One example of such a project is Ubergraph.

  Another limitation is that the script only works with the OBO format, both
  for its input and its output.}

\end{frame}

\begin{frame}
  \frametitle{Integration pipeline}

  \begin{columns}
    \begin{column}{.44\textwidth}
      \begin{tikzpicture}[node distance=.5cm]
        \node [Entity] (uberon) {Uberon};
        \node [Process,below right=of uberon] (script) {Custom script};
        \node [Entity,above right=of script,align=left] (ssao) {Taxon-specific\\
                                                                ontologies};
        \node [Entity,below=of script] (cob) {Bridge files};
        \node [Process,below=of cob] (merge) {OWL merge};
        \node [Entity,below=of merge] (cm) {Multi-species ontology};
        \draw [->] (uberon.east) -| (script.north) (script.south) -- (cob.north);
        \draw [->] (cob) -- (merge) (merge) -- (cm);
        \draw [->] (uberon.south) |- (merge.west);
        \draw [->] (ssao.south) |- (merge.east);
      \end{tikzpicture}
    \end{column}
    \begin{column}{.55\textwidth}
      Limitations:
      \begin{itemize}
        \item Lack of flexibility
          \begin{itemize}
            \item Cannot integrate foreign mappings
            \item Cannot generate a single bridge file
          \end{itemize}
        \item Mappings not available independently
        \item Not generalizable to other ontologies
      \end{itemize}
    \end{column}
  \end{columns}

  \note{Finally, here is an outline of the entire integration process, what
  happens when we do a Uberon release. From the main Uberon source file, we
  create the bridge files using the aforementioned script, and then we merge
  together: (1) Uberon itself, (2) the bridge files, and (3) the
  species-specific ontologies, to produce the integrated, multi-species
  ontology.

  While the process looks superficially simple (though it is, in fact, quite
  complex, it's just that most of the complexity is hidden behind that
  script), it is also severely limited.

  It is inflexible, in that the only possible source of mappings to generate
  the bridge files are the xrefs in Uberon itself -- we cannot integrate
  mappings coming from another source (which means we cannot delegate the
  maintenance of the mappings to the species-specific ontologies). Also, we
  cannot generate a single bridge file on-demand -- the script can only be
  used to generate all the bridge files in one step. It does not allow to
  publish the mappings themselves.

  It cannot be applied to other ontologies. There might be other projects that
  need to integrate several ontologies together, using mappings to connect
  terms across ontologies, but those projects cannot benefit from the work
  done in Uberon.}

\end{frame}

\section{New approach}

\begin{frame}
  \frametitle{Principles}

  \begin{enumerate}
    \item Clear separation between
      \begin{itemize}
        \item cross-ontology \emph{mappings} (COM)
        \item cross-ontology \emph{bridges} (COB)
      \end{itemize}
    \item Represent COM using a standard format intended for that purpose
      \begin{itemize}
        \item allowing for mapping-specific metadata
        \item allowing for meaningful mapping predicates
      \end{itemize}
    \item Make it easy to derive COB from COM
      \begin{itemize}
        \item without requiring Uberon-specific tooling
      \end{itemize}
  \end{enumerate}

  \note{Here are the key principles that dictated the design of the new
  approach.

  First, we want a clear separation between the \emph{mappings} and the
  \emph{bridges} that are derived from the mappings. In our current system,
  the mappings only exist as annotations within the ontology, and are not
  really treated as ``first-class objects''. In our new system, we want to
  focus on the mappings rather than the bridges, because the mappings are the
  actual informations that we are maintaining -- the bridges are merely a
  derivatiom from them. In particular, we want to be able to publish the
  mappings themselves, independently of the bridges.

  Second, we want to store the mappings using a format that is both standard
  and specifically intended for representing mappings. That format should,
  among other things, make it possible to store mapping-specific metadata (who
  asserted the mapping, etc.), make it possible to give precise meaning to a
  mapping (not just saying that two terms are mapped, but what is the nature
  of the link between them).

  Lastly, we want a bridge generation process that is much simpler than the
  current one, and that is not tied to Uberon in any way, so that it can be
  reused by other projects that have a similar need for integrating several
  ontologies together.}

\end{frame}

\begin{frame}
  \frametitle{The SSSOM standard}

  ``Simple Standard for Sharing Ontological Mappings''
  \begin{itemize}
    \item A data model to represent semantic mappings and associated metadata
    \item A simple TSV-based format to store and exchange them
  \end{itemize}

  \begin{block}{SSSOM/TSV format}\tiny\ttfamily
    \begin{tabular}{lllll}
      \hline
      subject\_id   & subject\_label                & predicate\_id                 & object\_id      & mapping\_justification\\
      \hline
      FBbt:00000001 & organism                      & semapv:crossSpeciesExactMatch & UBERON:0000468  & semapv:ManualMappingCuration\\
      FBbt:00000002 & tagma                         & semapv:crossSpeciesExactMatch & UBERON:6000002  & semapv:ManualMappingCuration\\
      ...           & ...                           & ...                           & ...             & ...\\
      FBbt:00003154 & heart                         & semapv:crossSpeciesExactMatch & UBERON:0015230  & semapv:ManualMappingCuration\\
      ...           & ...                           &...                           & ...             & ...\\
      \hline
    \end{tabular}
  \end{block}

  \note{Over the past three years, independently of Uberon, a new standard has
  been devised, specifically for the purpose of representing semantic
  mappings: the Simple Standard for Sharing Ontological Mappings.

  The SSSOM standard defines a common data model (formalised in LinkML) to
  represent and manipulate mappings in programming languages. The model
  notably allows to capture a rich set of metadata about each mapping.

  The standard also defines several serialisation formats to store and
  exchange mappings, the most important of which being the ``SSSOM/TSV''
  format, which as the name suggests is based on the ubiquitous TSV format,
  and is intended to make reusing mappings across resources as easy as
  possible.

  You see here an very basic example of what a mapping set in SSSOM/TSV format
  looks like: one mapping per line, each mapping consisting, as its core, of a
  subject, predicate, object triple. In this example, for simplicitly we only
  show one bit of metadata (the mapping justification).}

\end{frame}

\begin{frame}
  \frametitle{New mapping predicates for cross-species mappings}

  Additions to the \emph{Semantic Mapping Vocabulary} (SEMAPV)
  \begin{center}
    \begin{tikzpicture}
      \graph[tree layout,nodes={Term},edges={Relation}] {
        "skos:mappingRelation" <- {
          "semapv:isomorphicMatch" <- {
            "skos:exactMatch",
            "semapv:crossSpeciesExactMatch" [fill=GenericHighlight]
          },
          "semapv:nonIsomorphicMatch" <- {
            "skos:broadMatch" <- "semapv:crossSpeciesBroadMatch" [fill=GenericHighlight],
            "skos:narrowMatch" <- "semapv:crossSpeciesNarrowMatch" [fill=GenericHighlight]
          }
        };
      };
    \end{tikzpicture}
  \end{center}

  \note{One of the benefits of the SSSOM standard is that it does not simply
  allow to say that two terms are mapped, or somehow related, to each other.
  As we've seen in the last example, it also allows to specify an explicit
  mapping predicate, so as to give a precise meaning to each mapping. It is
  then possible to distinguish, for example between mappings of terms that
  represent exactly the same concept in two different vocabularies, and
  mappings of terms that represent \emph{similar}, but not exactly identical,
  concepts.

  In fact, as part of the efforts around the SSSOM standard, we have compiled
  a dedicated vocabulary, the \emph{Semantic Mapping Vocabulary}, to host,
  among other things, various mapping predicates that can be used to describe
  mappings. In that vocabulary, as part of this project, we have added a
  handful of predicates to specifically represent \emph{cross-species
  mappings}.

  For example, \emph{crossSpeciesExactMatch}, which is intended for mappings
  between terms that represent the same concept, modulo a taxonomic
  restriction. We can use that predicate to explicitly say that the FBbt term
  for the fly heart represents exactly the same concept as the Uberon term for
  dorsal vessel heart, \emph{except} for the fact that the former is specific
  to flies.}

\end{frame}

\begin{frame}[t]
  \frametitle{From SSSOM sets to bridge files: SSSOM/T-OWL}

  A domain-specific language to transform SSSOM mappings into OWL axioms

  \only<1>{
    \begin{block}{Model of a mapping processing rule}
      \begin{center}
        \begin{tikzpicture}
          \node [text width=5cm,fill=GenericItem,draw] (filter) at (0,0)
                {\textbf{Filter}\\
                 Mapping $\rightarrow$ boolean\\
                 Decide whether the rule applies to a mapping};
          \node [text width=5cm,fill=GenericItem,draw] (preprocessor) at (6,1)
                {\textbf{Preprocessor}\\
                 Mapping $\rightarrow$ mapping\\
                 Modify a mapping on the fly};
          \node [text width=5cm,fill=GenericItem,draw] (generator) at (6,-1)
                {\textbf{Generator}\\
                 Mapping $\rightarrow$ OWL axiom\\
                 Derive an axiom from a mapping};
          \draw[->] (filter) -- (preprocessor);
          \draw[->] (filter) -- (generator);
        \end{tikzpicture}
      \end{center}
    \end{block}
  }
  \only<2>{
    \begin{block}{Inverting a mapping}\ttfamily\scriptsize
subject==UBERON:* -> invert();
    \end{block}

    \begin{block}{Generating a simple axiom}\ttfamily\scriptsize
predicate==skos:broadMatch -> create\_axiom("\%subject\_id SubClassOf: \%object\_id");
    \end{block}

    \begin{block}{Generating a more complex axiom}\ttfamily\scriptsize
subject==FBbt:*\\
\hspace{1cm} \&\& predicate==semapv:crossSpeciesExactMatch\\
\hspace{1cm} \&\& (mapping\_justification==semapv:ManualMappingCuration || confidence>=0.8)\\
->  create\_axiom("\%subject\_id EquivalentTo: \%object\_id and\\
\hspace{2cm} (BFO:0000050 some NCBITaxon:7227"));
    \end{block}
  }

  \note<1>{Now that we can store mappings in a dedicated format, we need a way
  to use those mappings when we want to integrate several ontologies together.
  And, as stated in our goals, we want to do that in a way that is, as much as
  possible, not specific to Uberon, but reusable by other projects as well.

  For that, we have designed, and we propose, a small domain-specific language
  specifically intended to process SSSOM mappings and turn them into arbitrary
  OWL axioms. We call that language \emph{SSSOM/T-OWL}, or ``SSSOM Transform
  to OWL''.

  The core concept of SSSOM/T-OWL is the mapping processing rule, which is
  made of two elements.

  First, a filter that takes a mapping and returns a boolean value, indicating
  whether the rule should apply to that mapping or not; the filter can examine
  all the metadata of a mapping to make its decision.

  Second, either a preprocessor that takes a mapping and returns a modified
  version of that mapping (allowing to modify a mapping on the fly), or a
  generator that takes a mapping and returns an OWL axiom.}

  \note<2>{To illustrate, here are some examples of SSSOM/T-OWL rules.

  First, a simple rule with a simple filter that selects all mappings where
  the subject is a Uberon term, and apply an inversion preprocessor to them.
  Inversion flips a mapping around the predicate, so that the subject becomes
  the object and the other way round. So after applying this rule, all
  mappings with a Uberon subject in the set will become mappings with a Uberon
  object.

  Second, a rule with an equally simple filter that selects all mappings that
  use a \emph{skos:broadMatch} predicate, and generate from them a SubClassOf
  axiom that states that the subject term is a subclass of the object term
  (which represents a broader concept).

  Third, a more complex rule, with a compound filter that selects mappings
  that: (1) have a FBbt subject, (2) use a \emph{crossSpeciesExactMatch}
  predicate, (3) have either been manually curated, or have a confidence level
  higher than 80\%. When such a mapping is found, the rule creates an
  equivalence axiom stating that the subject (the FBbt term) is equivalent to
  the object that is part of the \emph{Drosophila melanogaster} taxon.}

\end{frame}

\begin{frame}
  \frametitle{Implementation in ROBOT commands}

  \begin{block}{sssom:xref-extract command}
    Convert existing mappings represented as cross-references into a SSSOM set

    \medskip
    {\ttfamily\scriptsize
robot sssom:xref-extract -i input.owl --mapping-file output.sssom.tsv
    }
  \end{block}

  \begin{block}{sssom:inject command}
    Apply SSSOM/T-OWL rules to a mapping set and inject the resulting axioms into an ontology

    \medskip
    {\ttfamily\scriptsize
robot sssom:inject -i input.owl --sssom mappings.sssom.tsv --ruleset bridge.rules
    }
  \end{block}

  \note{To actually implement our new approach, we first, as a preliminary
  step, added plugin support to ROBOT, the main tool we use to manipulate OWL
  and OBO ontologies. This is so that we can quickly add new commands to
  ROBOT. We then developed a SSSOM plugin for ROBOT, which provides two
  distinct commands.

  The first is a command to facilitate the transition to SSSOM-maintained
  mappings by automatically extracting mappings stored as cross-references
  within an ontology into a mapping set in the SSSOM/TSV format.

  More importantly, the second is a command that implements the SSSOM/T-OWL
  domain-specific language. That command takes a mapping set and a set of
  SSSOM/T-OWL rules as input, and will inject into the current ontology the
  axioms produced by applying the SSSOM/T-OWL rules to the mapping set.}

\end{frame}

\begin{frame}
  \frametitle{New SSSOM-based integration pipeline}

  \begin{center}
    \begin{tikzpicture}[node distance=.5cm]
      \node [Entity] (uberon) {Uberon};
      \node [Entity,right=of uberon] (com) {Mapping sets};
      \node [Entity,right=of com,align=left] (ssao) {Taxon-specific\\
                                                     ontologies};
      \node [Process,below=of com] (sssomt) {SSSOM/T-OWL};
      \node [Entity,below=of sssomt] (cob) {Bridge files};
      \node [Process,below=of cob] (merge) {OWL merge};
      \node [Entity,below=of merge] (cm) {Multi-species ontology};
      \draw [->] (com) -- (sssomt) (sssomt) -- (cob);
      \draw [->] (cob) -- (merge) (merge) -- (cm);
      \draw [->] (uberon.south) |- (merge.west);
      \draw [->] (ssao.south) |- (merge.east);
    \end{tikzpicture}
  \end{center}

  \note{This leads us to our now, SSSOM-based integration pipeline for Uberon,
  which now has three distinct inputs: (1) Uberon itself, (2) the mappings
  between Uberon and the taxon-specific ontologies, and (3) the taxon-specific
  ontologies.

  The mappings are transformed at build-time into OWL bridge files using a
  custom SSSOM/T-OWL ruleset, that contains rules similar to the one we have
  seen earlier: a rule that says that cross-species mappings between FBbt and
  Uberon are to be transformed into equivalence axioms with a taxonomic
  restriction to the \emph{Drosophila melanogaster} taxon, a rule that says
  that cross-species mappings between XAO and Uberon are to be transformed
  into equivalence axioms with a taxonomic restriction to the \emph{Xenopus}
  taxon, etc.

  After this step, we have only OWL files, which can all be merged using a
  standard ROBOT merge operation, producing the multi-species ontology that we
  want.}

\end{frame}

\section{Conclusion}

\begin{frame}
  \frametitle{Benefits of the new approach}

  \begin{block}{Cross-ontology mappings as first-class artefacts}
    \begin{itemize}
      \item With their own metadata
      \item Can be maintained separately
      \item Available as a separate release product in a standard format
    \end{itemize}
  \end{block}

  \begin{block}{Generic framework for SSSOM-based integration of multiple ontologies}
    \begin{itemize}
      \item SSSOM/T-OWL is completely independent of Uberon
      \item Only the ruleset is specific
    \end{itemize}
  \end{block}

  \note{For Uberon itself, the main benefit of the new approach is that we now
  have a pipeline that is both more flexible --~for example, it allows
  combining mapping sets from different sources as input~-- and more robust.

  For downstream users of Uberon, the mappings between Uberon and the
  taxon-specific ontologies are now first-class artefacts. They can have their
  own metadata, allowing for traceability and accountability of each
  individual mapping statement. They can be maintained separately from the
  Uberon ontology itself, for example to delegate them to the taxon-specific
  ontologies themselves (something we have already started doing for the
  mappings with FBbt, which are now maintained by FBbt rather than by Uberon).
  They are available as a distinct release product, which is findable on its
  own, in a format that is still relatively new but that we hope will become
  the standard format for semantic mappings.

  And for everyone else, we now have a generic framework to integrate multiple
  ontologies with SSSOM-maintained mappings. The SSSOM/T-OWL language, and its
  implementation in the SSSOM ROBOT plugin, has no ties whatsoever to Uberon
  and can be reused by any project. Only the SSSOM/T-OWL rules that we use are
  tailored to the needs of Uberon.}

\end{frame}

\begin{frame}[fragile]
  \frametitle{Example: FBbt/Uberon/CL integration test}

  A single ROBOT command to merge FBbt with Uberon, CL, and the mappings-derived
  bridging axioms, and check for possible inconsistencies that would reveal
  incorrect mappings:

  \begin{block}{}
\begin{verbatim}
robot merge -i fbbt.owl -i uberon.owl -i cl.owl \
      sssom:inject --sssom mappings.sssom.tsv   \
                   --ruleset bridging.rules     \
      reason -r ELK
\end{verbatim}
  \end{block}

  \note{Lastly, here is an example to illustrate some of the benefits of the
  approach for third-party users. As I've mentioned, at FlyBase we are now
  maintaining ourselves the mappings between FBbt and Uberon.

  Because FBbt and Uberon are developed separately, there is always a risk
  that, whenever something changes in either FBbt, Uberon, or the
  FBbt-to-Uberon mappings, an inter-ontology inconsistency is introduced.
  Testing to detect and prevent such inconsistencies is difficult and,
  historically, has never actually been done on a regular basis.

  But with the SSSOM approach, such testing can be done with a single ROBOT
  command in three steps, as shown here: (1) we merge the current versions of
  FBbt, Uberon, and CL; (2) we derive OWL axioms from the FBbt-to-Uberon/CL
  mappings and inject them into the merged ontology; (3) we reason over the
  merged ontology to detect inconsistencies. This takes but a couple of
  minutes, and that test is now part of our test suite in FBbt. Thanks to this
  approach, over the past two years we have rooted out all existing
  inconsistencies between FBbt and Uberon.}

\end{frame}

\end{document}

