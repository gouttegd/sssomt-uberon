% SSSOM/T and Uberon presentation
% © 2024 Damien Goutte-Gattat
% 
% This work is released under the Creative Commons Attribution 4.0
% International License. To view a copy of this license, visit
% https://creativecommons.org/licenses/by/4.0/.
\usepackage{fontspec}
\usepackage{polyglossia}
\usepackage{tikz}

\setmainlanguage{english}
\usetheme{cambridge}

\graphicspath{{imgs/}{svgs/}}

% TikZ libraries and styles
\usetikzlibrary{positioning,shapes.geometric,graphs,graphdrawing,quotes,fit,overlay-beamer-styles}
\usegdlibrary{layered,trees}
\colorlet{GenericItem}{yellow!20!white}
\colorlet{GenericHighlight}{red!20!white}
\tikzset{Term/.style={draw,fill=yellow!20!white,font=\tiny\bfseries}}
\tikzset{Relation/.style={draw=blue!40!white,thick,arrows=<-}}
\tikzset{RelationLabel/.style={shape=circle,fill=white,draw=black,thin,font=\tiny\bfseries,inner sep=1pt}}
\tikzset{File/.style={draw,fill=yellow!20!white,font=\tiny\ttfamily}}
\tikzset{Entity/.style={draw,fill=yellow!20!white,font=\scriptsize}}
\tikzset{Process/.style={draw,shape=ellipse,fill=blue!20!white,font=\scriptsize}}

\title{(Re-)bridging the anatomy ontologies with SSSOM}

\author{Damien~Goutte-Gattat\inst{1}}
\institute{\inst{1}FlyBase group, Department of Physiology, Development and Neuroscience}
\date{International Conference on Biomedical Ontologies 2024, July 2024}

\hypersetup{pdfpagemode=UseNone,
  pdftitle={},
  pdfauthor={Damien Goutte-Gattat}}
\AtBeginSection{
  \begin{frame}{Outline}{}
    \tableofcontents[currentsection]
  \end{frame}
}
\AtBeginSubsection{
  \begin{frame}{Outline}{}
    \tableofcontents[currentsection,currentsubsection]
  \end{frame}
}

\begin{document}

\begin{frame}
  \titlepage
\end{frame}

\section{Introduction}

\begin{frame}
  \frametitle{An approach to a multi-species anatomy ontology}

  \begin{columns}
    \begin{column}{.5\textwidth}
      \begin{tikzpicture}[nodes={shape=circle,draw,fill=GenericItem}]
        \node (core) [alt=<6| handout:0>{fill=GenericHighlight}{}] {Uberon};
        \node (fbbt) [alt=<{1,7| handout:0}>{fill=GenericHighlight}{}] at (150:2cm) {FBbt};
        \node (zfa)  [alt=<{2,7| handout:0}>{fill=GenericHighlight}{}] at (90:2cm)  {ZFA};
        \node (xao)  [alt=<{3,7| handout:0}>{fill=GenericHighlight}{}] at (30:2cm)  {XAO};
        \node (ma)   [alt=<{4,7| handout:0}>{fill=GenericHighlight}{}] at (210:2cm) {MA};
        \node (wbbt) [alt=<{5,7| handout:0}>{fill=GenericHighlight}{}] at (270:2cm) {WBbt};
        \node (tx1) at  (310:2cm) {...};
        \node (tx2) at  (330:2cm) {...};
        \node (tx3) at  (350:2cm) {...};
        \draw (core) -- (fbbt) (core) -- (zfa)  (core) -- (xao)
              (core) -- (ma)   (core) -- (wbbt) (core) -- (tx1)
              (core) -- (tx2)  (core) -- (tx3);
      \end{tikzpicture}
    \end{column}
    \begin{column}{.45\textwidth}
      Many species-specific anatomy ontologies available:
      \smallskip
      \begin{overlayarea}{\textwidth}{1cm}\color{red}
        \only<1>{\emph{Drosophila} Anatomy Ontology}
        \only<2| handout:0>{Zebrafish Anatomy Ontology}
        \only<3| handout:0>{Xenopus Anatomy Ontology}
        \only<4| handout:0>{Mouse Anatomy Ontology}
        \only<5| handout:0>{\emph{C. elegans} Anatomy Ontology}
      \end{overlayarea}

      Strategy for a multi-species ontology:
      \begin{itemize}
        \item Provide a \textcolor<6>{red}{taxon-neutral core framework}
        \item Integrate \textcolor<7>{red}{existing taxon-specific ontologies}
      \end{itemize}
    \end{column}
  \end{columns}

  \note<1>{Anatomy ontologies are used in many aspects of biocuration, for
  example to capture the organ of origin of a biological sample, the tissue
  where a gene is expressed, or the anatomical location where a disease
  manifests itself.

  Many species-specific curation projects have, therefore, their own ontology
  to describe the anatomy of their favourite organism. For example, at FlyBase
  we have our own \emph{Drosophila} Anatomy Ontology (FBbt), ZFIN has its
  Zebrafish Anatomy Ontology (ZFA), XenBase has its Xenopus Anatomy Ontology
  (XAO), and so on.

  But to enable \emph{cross-species} curation and analysis of data (especially
  high-throughput data, such as scRNAseq), we need ideally a single,
  multi-species ontology that covers all our model organisms. For the past 12
  years or so, this has been the goal of Uberon, the Uber Anatomy Ontology.

  Uberon's strategy to achieve that goal is to provide a taxon-neutral core
  framework, describing common anatomical structures in broadly generic terms,
  and to plug the species-specific ontologies into that core.}

\end{frame}

\begin{frame}
  \frametitle{Example: integrating FBbt into Uberon}

  \only<1| handout:1>{
    \begin{columns}[t]
      \begin{column}{.45\textwidth}
        Uberon hierarchy
        \begin{tikzpicture}
          \graph[layered layout,orient=down,nodes={Term},edge quotes mid,edges={Relation,nodes={RelationLabel}}] {
            anatomical entity <-
            anatomical structure <-
            circulatory organ <- {
              primary circulatory organ <- { heart, dorsal vessel heart[fill=GenericHighlight] },
              accessory circulatory organ <- lymph heart
            };
          };
        \end{tikzpicture}
      \end{column}
      \begin{column}{.45\textwidth}
        FBbt hierarchy
        \begin{tikzpicture}
          \graph[layered layout,orient=down,nodes={Term},edge quotes mid,edges={Relation,nodes={RelationLabel}}] {
            anatomical entity <-
            anatomical structure <-
            heart[fill=GenericHighlight] <-["P"] {
              aortic funnel, heart chamber, dorsal diaphragm
            };
          };
        \end{tikzpicture}

        FBbt's ``heart'' is the fly-specific equivalent of Uberon's ``dorsal vessel heart''
      \end{column}
    \end{columns}
  }
  \only<2| handout:2>{
    \begin{columns}
      \begin{column}{.3\textwidth}
        Integrated hierarchy:

        \smallskip
        FBbt's ``heart'' classified under Uberon's ``dorsal vessel heart''
      \end{column}
      \begin{column}{.6\textwidth}
        \begin{tikzpicture}
          \graph[layered layout,orient=down,nodes={Term},edge quotes mid,edges={Relation,nodes={RelationLabel}}] {
            anatomical entity <-
            anatomical structure <-
            circulatory organ <- {
              primary circulatory organ <- {
                heart,
                dorsal vessel heart[fill=GenericHighlight]
                    <- "heart (drosophila)"[fill=GenericHighlight] <-["P"] {
                  aortic funnel, heart chamber, dorsal diaphragm
                }
              },
              accessory circulatory organ <- lymph heart
            };
          };
        \end{tikzpicture}
      \end{column}
    \end{columns}
  }
  
  \note<1>{To illustrate this strategy, here is (on the left) a simplified
  excerpt of Uberon's hierarchy around the concept of ``heart'', which in
  Uberon parlance is called, more generically, the ``primary circulatory
  organ'' (to make room for the \emph{accessory} circulatory organs that can
  be found in some species).

  Below ``primary circulatory organ'', Uberon defines both ``heart'', which is
  intended to refer to the \emph{vertebrate} heart, and ``dorsal vessel
  heart'', which refers to the heart of arthropods such as fruit flies. Uberon
  itself does not describe the arthropod heart in any more details -- there
  are no terms under ``dorsal vessel heart''. That term is in fact merely
  intended as a ``branching point'', where more specific terms from
  arthropod-specific ontologies can be rattached.

  In the \emph{Drosophila} Anatomy Ontology, we have a single term ``heart''
  to refer to the dorsal vessel heart (we don't need to call it ``dorsal
  vessel heart'' since it is the only heart that exists in flies), and below
  that term, we have a handful of terms that describe more precisely the
  different parts that make up the fly heart.}

  \note<2>{When FBbt is integrated into Uberon, its ``heart'' term is
  automatically treated as a subclass of the more generic ``dorsal vessel
  heart'', thereby creating a unified hierarchy where the
  \emph{Drosophila}-specific terms are appropriately rattached to Uberon's
  taxon-neutral core.

  Using this strategy, Uberon is able to accurately and precisely describe the
  anatomy of the most used model organisms, by benefitting from all the
  existing species-specific ontologies without having to duplicate their
  efforts.

  This talk is about how this integration process works, and how we recently
  overhauled it to make it easier to maintain, more robust, and FAIRer.}

\end{frame}

\begin{frame}
  \frametitle{Integration requires cross-ontology mappings}

  \begin{block}{}
    Integrating the taxon-specific ontologies with the Uberon core framework
    requires to know where taxon-specific terms should be ``plugged'' in the
    core framework.

    E.g., we need to know that FBbt's \emph{heart} must be attached to Uberon's
    \emph{dorsal vessel heart}.
  \end{block}

  \note{The key component of the integration process is the knowledge of which
  species-specific terms, from each of the species-specific ontologies such as
  FBbt, should be ``attached'' to which taxon-neutral term in Uberon's core
  framework. For example, knowing that FBbt's ``heart'' is the fly-specific
  term for what Uberon calls the ``dorsal vessel heart''.

  This is the role of cross-ontology \emph{mappings}.}

\end{frame}

\section{Existing system and its limitations}

\begin{frame}
  \frametitle{Integration pipeline overview}

  \begin{columns}
    \begin{column}{.46\textwidth}
      \begin{tikzpicture}[node distance=.3cm]
        \node [Entity,text width=1.3cm,alt=<1>{fill=GenericHighlight}{}] (uberon) {Uberon\\(incl. xrefs)};
        \node [Process,below right=of uberon] (script) {Custom script};
        \node [Entity,above right=of script,align=left] (ssao) {Taxon-specific\\ontologies};
        \node [Entity,below=of script,alt=<2>{fill=GenericHighlight}{}] (cob) {Bridge files};
        \node [Process,below=of cob] (merge) {OWL merge};
        \node [Entity,below=of merge] (cm) {Multi-species ontology};
        \draw [->] (uberon.east) -| (script.north) (script.south) -- (cob.north);
        \draw [->] (cob) -- (merge) (merge) -- (cm);
        \draw [->] (uberon.south) |- (merge.west);
        \draw [->] (ssao.south) |- (merge.east);
      \end{tikzpicture}
    \end{column}
    \begin{column}{.53\textwidth}
      \begin{block}{Mappings maintained as \textcolor<1>{red}{xrefs}}\scriptsize\ttfamily
[Term]

id: UBERON:0015230

name: dorsal vessel heart

is\_a: UBERON:0007100 ! primary circulatory organ

\textcolor<1>{red}{xref: FBbt:00003154}
      \end{block}

      \begin{block}{Xrefs transformed to \textcolor<2>{red}{bridge files}}\scriptsize\ttfamily
[Term]

id: FBbt:00003154

intersection\_of: UBERON:0015230

intersection\_of: part\_of NCBITaxon:7227
      \end{block}
    \end{column}
  \end{columns}

  \note<1>{Here is a brief overview of the integration pipeline.

  Mappings between Uberon and the taxon-specific ontologies are maintained as
  cross-references that are carried by the taxon-neutral terms in Uberon and
  that point to the taxon-specific terms in other ontologies -- as shown here,
  where Uberon's ``dorsal vessel heart'' term carries a ``xref'' to
  FBbt:0003154, which is FBbt's term for the fly heart.

  At build time, xrefs are extracted by a custom script that turns them into
  bridge files, which contains OBO materialisations of the mappings, in the
  form of equivalence axioms (such as an axiom stating that FBbt's ``heart''
  is equivalent to a Uberon ``dorsal vessel heart'' that is part of the
  \emph{Drosophila melanogaster} taxon).

  While the process looks superficially simple (though it is, in fact, quite
  complex, it's just that most of the complexity is hidden behind that
  script), it is also severely limited.

  The limitations stem both from the way the mappings are represented, and the
  way they are transformed into bridge files.}

\end{frame}

\begin{frame}
  \frametitle{Issues with mappings maintained as cross-references}

  \begin{block}{Mappings embedded within the ontology}
    \begin{itemize}
      \item Not available separately
     \end{itemize}
  \end{block}

  \begin{block}{No additional data}
    \begin{itemize}
      \item All the cross-references say is that the mappings exist
    \end{itemize}
  \end{block}

  \begin{block}{Meaningless mapping predicate}
    \begin{itemize}
      \item Overloading of the \emph{oboInOwl:hasDbXref} property
      \item Workaround: {\ttfamily\scriptsize treat-xrefs-as-reverse-genus-differentia FBbt NCBITaxon:7227}
        \begin{itemize}
          \item Assigns a meaning to \emph{all} cross-references between FBbt and Uberon
        \end{itemize}
    \end{itemize}
  \end{block}

  \note{The one benefit of using xrefs to represent mappings is that it is a
  very simple method, but that simplicity comes at a high cost.

  First, the mappings are embedded within the ontology and as such are not
  independently available -- they need to be forcefully extracted whenever
  someone needs them. This makes the mappings more difficult to find, access,
  and reuse than it should be.

  Second, the cross-reference is telling us that there is a mapping, but we
  can't know anything else about that mapping. Why does that mapping exist?
  Who asserted it? When? On what basis? This is simply not recorded.

  Third, the \emph{oboInOwl:hasDbXref} annotation has been used for a lot of
  things since its introduction, and no longer has a clear and unambiguous
  meaning -- what it means depends on where it is used. At best, it indicates
  that two terms are mapped \emph{somehow}, but the nature of the relation is
  unspecified.

  In Uberon we can circumvent that last issue by using supplementary
  ontology-level annotations to provide meaning to some cross-references. But
  this is more a workaround than a proper solution, and it comes with problems
  of its own.}

\end{frame}

\begin{frame}
  \frametitle{Issues with the bridge generation process}

  \begin{columns}
    \begin{column}{.4\textwidth}
      \begin{center}
        \begin{tikzpicture}
          \node [File] (uberon) at (0,0)    {uberon.obo};
          \node [File] (script) at (0,-1)   {make-bridge-ontology-from-xrefs.pl};
          \node [File] (bridge) at (0,-2)   {uberon-bridge-to-fbbt.obo};
          \node [File]          at (0,-2.5) {uberon-bridge-to-wbbt.obo};
          \node [File]          at (0,-3)   {uberon-bridge-to-ma.obo};
          \node [File]          at (0,-3.5) {uberon-bridge-to-....obo};
          \draw[->] (uberon) -- (script) (script) -- (bridge);
        \end{tikzpicture}
      \end{center}
    \end{column}
    \begin{column}{.55\textwidth}
      \begin{itemize}
        \item Lack of flexibility
        \item Maintenance nightmare
        \item Dependent on the OBO format
        \item Highly Uberon-specific
      \end{itemize}
    \end{column}
  \end{columns}

  \note{The bridge files are generated by this custom Perl script, which reads
  the Uberon ontology file and generates one bridge file for each
  taxon-specific ontology. That script is plagued with several issues.

  The process is very rigid. For example, it is impossible to generate
  only one bridge file on-demand. Almost everything is hardcoded in the
  script, making it difficult to change anything whenever needed.

  The script is difficult to maintain for most Uberon editors and developers,
  in no small part because it conflates the technical aspects of extracting
  the cross-references and generating the bridges, and the logical aspects of
  determining which bridging axioms should be produced.

  It's a smaller issue, but the script also only works with the OBO files,
  both for its input and its output.

  But the main problem is that the bridge generation script is highly specific
  to Uberon. It is almost impossible to reuse its logic elsewhere. Yet the
  need for integrating several ontologies together, using mappings as glue to
  connect terms between those ontologies, is certainly not a Uberon-specific
  need.}

\end{frame}

\begin{frame}
  \frametitle{Goals for a new approach}
 
  \begin{enumerate}
    \item Treat cross-ontology mappings as proper entities
      \begin{itemize}
        \item Mappings set made available as independent artefact
      \end{itemize}
    \item Represent COMs using a standard format intended for that purpose
      \begin{itemize}
        \item allowing for mapping-specific metadata
        \item allowing for meaningful mapping predicates
      \end{itemize}
    \item Make it easy to derive bridging axioms from mappings
      \begin{itemize}
        \item without requiring Uberon-specific tooling
      \end{itemize}
  \end{enumerate}

  \note{In response to the limits of the previous system, we set out to design
  a new system, with the following aims:

  First, we want the cross-ontology mappings to be treated as first-class
  objects, similarly to the terms of the ontology, rather than merely
  annotations on the terms. They should in fact be treated as the source
  artefact from which the bridge files are derived. As with other source
  artefacts, they must be published on their own and be independently findable.

  Second, we want to store the mappings using a format that is both standard
  and specifically intended for representing mappings. That format should make
  it possible to store mapping-specific metadata, and to give precise meaning
  to every individual mapping (everything that is not possible with the
  cross-reference format).

  Lastly, we want a bridge generation process that is much simpler than the
  current one, and that is not tied to Uberon in any way, so that it can be
  reused by other projects that have a similar need for integrating several
  ontologies together.}

\end{frame}

\section{Implementation of a new approach}

\subsection{New representation of mappings}

\begin{frame}
  \frametitle{The SSSOM standard}

  ``Simple Standard for Sharing Ontological Mappings''
  \begin{itemize}
    \item A data model to represent semantic mappings and associated metadata
    \item A simple TSV-based format to store and exchange them
  \end{itemize}

  \begin{block}{SSSOM/TSV format}\tiny\ttfamily
    \begin{tabular}{lllll}
      \hline
      subject\_id   & predicate\_id                 & object\_id      & mapping\_justification        & mapping\_date\\
      \hline
      FBbt:00000001 & semapv:crossSpeciesExactMatch & UBERON:0000468  & semapv:ManualMappingCuration  & 2024-06-01\\
      FBbt:00000002 & semapv:crossSpeciesExactMatch & UBERON:6000002  & semapv:ManualMappingCuration  & 2024-06-01\\
      ...           & ...                           & ...             & ...                           & ...\\
      \textcolor<2>{red}{FBbt:00003154} & \textcolor<2>{red}{semapv:crossSpeciesExactMatch} & \textcolor<2>{red}{UBERON:0015230}  & \textcolor<3>{red}{semapv:ManualMappingCuration}  & \textcolor<3>{red}{2024-06-01}\\
      ...           & ...                           & ...             & ...                           & ...\\
      \hline
    \end{tabular}
  \end{block}

  \begin{overlayarea}{\textwidth}{1cm}
    \only<2>{Mapping proper: subject -- predicate -- object triple}
    \only<3>{Mapping metadata}
  \end{overlayarea}

  \note<1>{Over the past three years, independently of Uberon, a new standard
  has been devised to work with semantic mappings: the Simple Standard for
  Sharing Ontological Mappings.

  The SSSOM standard defines a common data model to represent and manipulate
  mappings in programming languages. The model notably allows to capture a
  rich set of metadata about each mapping.

  The standard also defines several serialisation formats to store and
  exchange mappings, the most important of which being the ``SSSOM/TSV''
  format, intended to make reusing mappings across resources as easy as
  possible.

  You see here an very basic example of what a mapping set in SSSOM/TSV format
  looks like: one mapping per line, each mapping consisting, at its core, of a
  triple made of a subject, an object, and a mapping predicate. That core is
  associated with several metadata.}

\end{frame}

\begin{frame}
  \frametitle{Meaningful predicates for cross-species mappings}

  Additions to the \emph{Semantic Mapping Vocabulary} (SEMAPV)
  \begin{center}
    \begin{tikzpicture}
      \graph[tree layout,nodes={Term},edges={Relation}] {
        "skos:mappingRelation" <- {
          "semapv:isomorphicMatch" <- {
            "skos:exactMatch",
            "semapv:crossSpeciesExactMatch" [fill=GenericHighlight]
          },
          "semapv:nonIsomorphicMatch" <- {
            "skos:broadMatch" <- "semapv:crossSpeciesBroadMatch" [fill=GenericHighlight],
            "skos:narrowMatch" <- "semapv:crossSpeciesNarrowMatch" [fill=GenericHighlight]
          }
        };
      };
    \end{tikzpicture}
  \end{center}

  \emph{crossSpeciesExactMatch}: ``a match where the subject is considered analogous
  to the object in a different taxonomic grouping''

  \note{One of the benefits of the SSSOM model is that it allows to specify
  an explicit mapping predicate, so as to give a precise meaning to each
  mapping. It is then possible to distinguish, for example between mappings of
  terms that represent exactly the same concept in two different vocabularies,
  and mappings of terms that represent \emph{similar}, but not exactly
  identical, concepts.

  The authors of the SSSOM standard have compiled a dedicated vocabulary, the
  \emph{Semantic Mapping Vocabulary}, to host, among other things, various
  mapping predicates that can be used to qualify mappings. In that vocabulary,
  as part of this project, we have added a handful of predicates to
  specifically represent \emph{cross-species mappings}.

  The most important being \emph{crossSpeciesExactMatch}, which is intended
  for mappings between terms that represent the same concept, modulo a
  taxonomic restriction. We can use that predicate to explicitly say that the
  FBbt term for the fly heart represents exactly the same concept as the
  Uberon term for dorsal vessel heart, \emph{except} for the fact that the
  former is specific to flies.}

\end{frame}

\subsection{New bridge generation process}

\begin{frame}[t]
  \frametitle{From SSSOM sets to bridge files: SSSOM/T-OWL}

  A domain-specific language to transform SSSOM mappings into OWL axioms

  \begin{block}{Model of a mapping processing rule}
    \begin{center}
      \begin{tikzpicture}
        \node [text width=5cm,fill=GenericItem,draw] (filter) at (0,0)
              {\textbf{Filter}\\
               Mapping $\rightarrow$ boolean\\
               Decide whether the rule applies to a mapping};
        \node [text width=5cm,fill=GenericItem,draw] (preprocessor) at (6,1)
              {\textbf{Preprocessor}\\
               Mapping $\rightarrow$ mapping\\
               Modify a mapping on the fly};
        \node [text width=5cm,fill=GenericItem,draw] (generator) at (6,-1)
              {\textbf{Generator}\\
               Mapping $\rightarrow$ OWL axiom\\
               Derive an axiom from a mapping};
        \draw[->] (filter) -- (preprocessor);
        \draw[->] (filter) -- (generator);
      \end{tikzpicture}
    \end{center}
  \end{block}

  \note{Now that we can store mappings in a dedicated format, along with their
  metadata and with meaningful mapping predicates, we need a way to use those
  mappings when we want to integrate several ontologies together.  And, as per
  our third goal, we want to do that in a way that is, as much as possible,
  not specific to Uberon, but reusable by other projects as well.

  For that, we have designed, and we propose, a small domain-specific language
  specifically intended to process SSSOM mappings and turn them into arbitrary
  OWL axioms. We call that language \emph{SSSOM/T-OWL}, or ``SSSOM Transform
  to OWL''.

  The core concept of SSSOM/T-OWL is the mapping processing rule, which is
  made of two elements.

  First, a filter that takes a mapping and returns a boolean value, indicating
  whether the rule should apply to that mapping or not; the filter can examine
  all the metadata of a mapping to make its decision.

  Second, either a preprocessor that takes a mapping and returns a modified
  version of that mapping (allowing to modify a mapping on the fly), or a
  generator that takes a mapping and returns an OWL axiom.}
\end{frame}

\begin{frame}
  \frametitle{From SSSOM sets to bridge files: SSSOM/T-OWL}

  Examples of SSSOM/T-OWL rules

  \begin{overlayarea}{\textwidth}{3cm}
    \only<1-2| handout:1>{
      \begin{block}{Inverting a mapping}\ttfamily\scriptsize
        \textcolor<1>{red}{subject==UBERON:*} -> \textcolor<2>{red}{invert()};
      \end{block}
    }
    \only<3-4| handout:2>{
      \begin{block}{Generating a simple axiom}\ttfamily\scriptsize
        \textcolor<3>{red}{predicate==skos:broadMatch} -> \textcolor<4>{red}{create\_axiom("\%subject\_id SubClassOf: \%object\_id")};
      \end{block}
    }
    \only<5-10| handout:3>{
      \begin{block}{Generating a more complex axiom}\ttfamily\scriptsize
        \textcolor<5>{red}{subject==FBbt:*}\\
        \hspace{1cm} \&\& \textcolor<6>{red}{predicate==semapv:crossSpeciesExactMatch}\\
        \hspace{1cm} \&\&
        (\textcolor<7>{red}{mapping\_justification==semapv:ManualMappingCuration}
        \textcolor<8>{red}{||} \textcolor<9>{red}{confidence>=0.8})\\
        ->  \textcolor<10>{red}{create\_axiom("\%subject\_id EquivalentTo: \%object\_id and\\
        \hspace{2cm} (BFO:0000050 some NCBITaxon:7227"))};
      \end{block}
    }
  \end{overlayarea}

  \begin{overlayarea}{\textwidth}{2.5cm}
    \only<1-2| handout:1>{\textcolor<1>{red}{Select mappings with a UBERON subject}

    and \textcolor<2>{red}{flip them around the predicate}}
    \only<3-4| handout:2>{\textcolor<3>{red}{Select mappings with a \emph{skos:broadMatch} predicate}

    and \textcolor<4>{red}{create a \emph{SubClassOf} axiom}}
    \only<5-10| handout:3>{Select mappings
      \begin{itemize}
        \item \textcolor<5>{red}{with a FBbt subject}
        \item \textcolor<6>{red}{with a \emph{semapv:crossSpeciesExactMatch} predicate}
        \item that have been \textcolor<7>{red}{manually curated} \textcolor<8>{red}{or} have a \textcolor<9>{red}{confidence higher than 80\%}
      \end{itemize}

      and \textcolor<10>{red}{create an equivalence axiom with a taxonomic restriction}}
  \end{overlayarea}

  \note<1>{To illustrate, here are some examples of SSSOM/T-OWL rules.

  First, a simple rule with a simple filter that selects all mappings where
  the subject is a Uberon term, and applies an inversion preprocessor to them.
  Inversion flips a mapping around the predicate, so that the subject becomes
  the object and the other way round. So after applying this rule, all
  mappings with a Uberon subject in the set will become mappings with a Uberon
  object.

  Second, a rule with an equally simple filter that selects all mappings that
  use a \emph{skos:broadMatch} predicate, and generate from them a SubClassOf
  axiom that states that the subject term is a subclass of the object term
  (which represents a broader concept).

  Third, a more complex rule, with a compound filter that selects mappings
  that: (1) have a FBbt subject, (2) use a \emph{crossSpeciesExactMatch}
  predicate, (3) have either been manually curated, or have a confidence level
  higher than 80\%. When such a mapping is found, the rule creates an
  equivalence axiom stating that the subject (the FBbt term) is equivalent to
  the object that is part of the \emph{Drosophila melanogaster} taxon.}

\end{frame}

\begin{frame}
  \frametitle{Adding SSSOM/T-OWL support to ROBOT}

  \begin{block}{Prerequisite step}
    Support for ``pluggable commands'' in ROBOT (since version 1.9.5)
  \end{block}

  \begin{block}{sssom:inject command}
    Apply \textcolor<2>{red}{SSSOM/T-OWL} rules to \textcolor<3>{red}{a
    mapping set} and inject the resulting axioms into the
    \textcolor<4>{red}{current ontology}

    \medskip
    {\ttfamily\scriptsize
robot sssom:inject -i \textcolor<4>{red}{input.owl} --sssom \textcolor<3>{red}{mappings.sssom.tsv} --ruleset \textcolor<2>{red}{bridge.rules}
    }
  \end{block}

  \note<1>{With the SSSOM/T-OWL language fully designed, we then needed to
  actually implement it. We chose to add support for SSSOM/T-OWL to ROBOT,
  since this is the main tool we use for most of our pipelines.

  The first step was to add plugin support for ROBOT -- that is, the
  possibility to add third-party commands to ROBOT. This is so that we could
  quickly add new commands, such as SSSOM-related commands, to ROBOT without
  being tied by ROBOT's release cycle. This was done with version 1.9.5 of
  ROBOT, released at the end of Summer 2023.

  We then developed a SSSOM plugin for ROBOT, which among other things,
  provides a command that implement our SSSOM/T-OWL language. That command
  will read a set of SSSOM/T-OWL rules and apply them to a set of mappings,
  thereby producing bridging axioms which will then be injected into the
  ontology that is currently being manipulated by ROBOT.}

\end{frame}

\begin{frame}
  \frametitle{New SSSOM-based integration pipeline}

  \begin{columns}
    \begin{column}{.45\textwidth}
      \begin{center}
        \begin{tikzpicture}[node distance=.5cm]
          \node [Entity] (uberon) {Uberon};
          \node [Entity,right=of uberon,alt=<2>{fill=GenericHighlight}{}] (com) {Mapping sets};
          \node [Entity,right=of com,align=left,alt=<3>{fill=GenericHighlight}{}] (rules) {SSSOM/T-OWL\\
                                                                                           ruleset};
          \node [Process,below=of com,alt=<4>{fill=GenericHighlight}{}] (inject) {sssom:inject};
          \node [Process,below=of inject,alt=<5>{fill=GenericHighlight}{}] (merge) {merge};
          \node [Entity,right=of merge,align=left] (ssao) {Taxon-specific\\
                                                         ontologies};
          \node [Entity,below=of merge] (cm) {Multi-species ontology};
          \draw [->] (com) -- (inject) (inject) -- (merge) (merge) -- (cm);
          \draw [->] (rules.south) |- (inject.east);
          \draw [->] (uberon.south) |- (inject.west);
          \draw [->] (ssao.west) -- (merge.east);
        \end{tikzpicture}
      \end{center}
    \end{column}
    \begin{column}{.45\textwidth}
      \begin{enumerate}
        \item Maintaining (and collecting) mappings as \textcolor<2>{red}{SSSOM sets}
        \item Maintaining the integration logic as \textcolor<3>{red}{SSSOM/T-OWL rules}
        \item Generating the bridging axioms through \textcolor<4>{red}{SSSOM/T-OWL support in ROBOT}
        \item \textcolor<5>{red}{Merging in} the taxon-specific ontologies
      \end{enumerate}
    \end{column}
  \end{columns}

  \note<1>{This leads us to our new, SSSOM-based integration pipeline for
  Uberon.

  The mappings are now maintained separately from Uberon itself, as SSSOM
  mapping sets.

  The integration logic, which decides how the bridging axioms should be
  generated, is also maintained separately, as a large ruleset written in the
  SSSOM/T-OWL language. It contains rules similar to the ``complex'' example
  of rule shown previously.

  At build time, we use our SSSOM ROBOT plugin to apply that ruleset to our
  mappings, and inject the resulting bridging axioms into Uberon.

  Then, we just need to merge the taxon-specific ontologies into the result of
  the previous step, producing the multi-species ontology that we want.

  Overall, that pipeline is both more robust, as most of the code is
  consolidated in the ROBOT plugin, and more flexible, as the decision logic
  is easily modifiable in the SSSOM/T-OWL ruleset.}

\end{frame}

\section{Conclusion}

\begin{frame}
  \frametitle{Benefits of the new approach}

  \begin{block}{Cross-ontology mappings as first-class artefacts}
    \begin{itemize}
      \item With their own metadata
      \item Can be maintained separately
      \item Available as a separate release product in a standard format
    \end{itemize}
  \end{block}

  \begin{block}{Generic framework for SSSOM-based integration of multiple ontologies}
    \begin{itemize}
      \item SSSOM/T-OWL is completely independent of Uberon
      \item Only the ruleset is specific
    \end{itemize}
  \end{block}

  \note{Beyond the more robust and more flexible pipeline, which is mostly of
  interest for Uberon editors and developers, the new approch has broader
  benefits.

  The mappings between Uberon and the taxon-specific ontologies are now
  first-class artefacts. They can have their own metadata, allowing for
  traceability and accountability of each individual mapping statement, and
  their own mapping predicate. They can be maintained separately from the
  Uberon ontology itself, and we can even delegate the maintenance of some of
  them to the taxon-specific ontologies themselves (something we have already
  started doing for the mappings with FBbt, which are now maintained by
  FlyBase rather than by Uberon).  They are available as a distinct release
  product, which is findable on its own, in a format that is still relatively
  new but that we hope will become the standard format for semantic mappings.

  And beyond Uberon, we now have a generic framework to integrate multiple
  ontologies with SSSOM-maintained mappings. The SSSOM/T-OWL language, and its
  implementation in the SSSOM ROBOT plugin, has no ties whatsoever to Uberon
  and can be reused by any project. Only the SSSOM/T-OWL rules that we use are
  tailored to the needs of Uberon.}

\end{frame}

\begin{frame}[fragile]
  \frametitle{Example of reuse: FBbt/Uberon integration test}

  A single ROBOT command to \textcolor<1>{red}{merge FBbt with Uberon},
  \textcolor<2>{red}{inject the mappings-derived bridging axioms}, and
  \textcolor<3>{red}{check for possible inconsistencies} that would reveal
  incorrect mappings:

  \begin{block}{}
\begin{semiverbatim}
robot \textcolor<1>{red}{merge} -i \textcolor<1>{red}{fbbt.owl} -i \textcolor<1>{red}{uberon.owl} \\
      \textcolor<2>{red}{sssom:inject} --sssom \textcolor<2>{red}{mappings.sssom.tsv}   \\
                   --ruleset \textcolor<2>{red}{bridging.rules}     \\
      \textcolor<3>{red}{reason} -r ELK
\end{semiverbatim}
  \end{block}

  \note<1>{Lastly, here is an example to illustrate some of the benefits of
  reusable integration logic.

  Because FBbt and Uberon are developed separately, there is always a risk
  that, whenever something changes in either FBbt or Uberon, an inter-ontology
  inconsistency is introduced.  Because the logic to integrate FBbt into
  Uberon was impossible to reuse outside of Uberon, it was difficult for FBbt
  editors to detect and prevent such inconsistencies whenever we edited the
  ontology.

  But with the SSSOM approach, such testing can be done with a single ROBOT
  command in three steps, as shown here: (1) we merge the current versions of
  FBbt and Uberon; (2) we derive OWL axioms from the FBbt-to-Uberon mappings
  and inject them into the merged ontology; (3) we reason over the merged
  ontology to detect inconsistencies. This takes but a couple of minutes, and
  that check is now part of our test suite in FBbt. Thanks to this approach,
  over the past two years we have rooted out all existing inconsistencies
  between FBbt and Uberon.}

\end{frame}

\begin{frame}
  \frametitle{About}

  \begin{block}{Acknowledgements}
    \begin{itemize}
      \item Grant BB/T014008 from BBSRC (UK) and NSF/BIO (US)
      \item Mapping Commons project for the SSSOM standard
    \end{itemize}
  \end{block}

  \begin{block}{\includegraphics[scale=.26]{ccby}\hspace{.2em}%
    2024 Damien Goutte-Gattat \texttt{<dpg44@cam.ac.uk>}}
    This presentation is released under the Creative Commons Attribution
    4.0 International License (CC BY 4.0).
  \end{block}

  \begin{block}{Revision}\ttfamily\scriptsize
    \input{revision.tex}
  \end{block}
\end{frame}

\appendix
\begin{frame}[noframenumbering]
  \frametitle{Questions?}
\end{frame}

\end{document}

% vim: tw=78 et ai ts=2 st=2 sw=2
